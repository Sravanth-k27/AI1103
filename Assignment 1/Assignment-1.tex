\documentclass[journal,12pt,twocolumn]{IEEEtran}

\usepackage{setspace}
\usepackage{gensymb}
\singlespacing
\usepackage[cmex10]{amsmath}

\usepackage{amsthm}

\usepackage{mathrsfs}
\usepackage{txfonts}
\usepackage{stfloats}
\usepackage{bm}
\usepackage{cite}
\usepackage{cases}
\usepackage{subfig}

\usepackage{longtable}
\usepackage{multirow}

\usepackage{enumitem}
\usepackage{mathtools}
\usepackage{steinmetz}
\usepackage{tikz}
\usepackage{circuitikz}
\usepackage{verbatim}
\usepackage{tfrupee}
\usepackage[breaklinks=true]{hyperref}
\usepackage{graphicx}
\usepackage{tkz-euclide}

\usetikzlibrary{calc,math}
\usepackage{listings}
    \usepackage{color}                                            %%
    \usepackage{array}                                            %%
    \usepackage{longtable}                                        %%
    \usepackage{calc}                                             %%
    \usepackage{multirow}                                         %%
    \usepackage{hhline}                                           %%
    \usepackage{ifthen}                                           %%
    \usepackage{lscape}     
\usepackage{multicol}
\usepackage{chngcntr}

\DeclareMathOperator*{\Res}{Res}

\renewcommand\thesection{\arabic{section}}
\renewcommand\thesubsection{\thesection.\arabic{subsection}}
\renewcommand\thesubsubsection{\thesubsection.\arabic{subsubsection}}

\renewcommand\thesectiondis{\arabic{section}}
\renewcommand\thesubsectiondis{\thesectiondis.\arabic{subsection}}
\renewcommand\thesubsubsectiondis{\thesubsectiondis.\arabic{subsubsection}}


\hyphenation{op-tical net-works semi-conduc-tor}
\def\inputGnumericTable{}                                 %%

\lstset{
%language=C,
frame=single, 
breaklines=true,
columns=fullflexible
}
\begin{document}

\newcommand{\BEQA}{\begin{eqnarray}}
\newcommand{\EEQA}{\end{eqnarray}}
\newcommand{\define}{\stackrel{\triangle}{=}}
\bibliographystyle{IEEEtran}
\raggedbottom
\setlength{\parindent}{0pt}
\providecommand{\mbf}{\mathbf}
\providecommand{\pr}[1]{\ensuremath{\Pr\left(#1\right)}}
\providecommand{\qfunc}[1]{\ensuremath{Q\left(#1\right)}}
\providecommand{\sbrak}[1]{\ensuremath{{}\left[#1\right]}}
\providecommand{\lsbrak}[1]{\ensuremath{{}\left[#1\right.}}
\providecommand{\rsbrak}[1]{\ensuremath{{}\left.#1\right]}}
\providecommand{\brak}[1]{\ensuremath{\left(#1\right)}}
\providecommand{\lbrak}[1]{\ensuremath{\left(#1\right.}}
\providecommand{\rbrak}[1]{\ensuremath{\left.#1\right)}}
\providecommand{\cbrak}[1]{\ensuremath{\left\{#1\right\}}}
\providecommand{\lcbrak}[1]{\ensuremath{\left\{#1\right.}}
\providecommand{\rcbrak}[1]{\ensuremath{\left.#1\right\}}}
\theoremstyle{remark}
\newtheorem{rem}{Remark}
\newcommand{\sgn}{\mathop{\mathrm{sgn}}}
\providecommand{\abs}[1]{\vert#1\vert}
\providecommand{\res}[1]{\Res\displaylimits_{#1}} 
\providecommand{\norm}[1]{\lVert#1\rVert}
%\providecommand{\norm}[1]{\lVert#1\rVert}
\providecommand{\mtx}[1]{\mathbf{#1}}
\providecommand{\mean}[1]{E[ #1 ]}
\providecommand{\fourier}{\overset{\mathcal{F}}{ \rightleftharpoons}}
%\providecommand{\hilbert}{\overset{\mathcal{H}}{ \rightleftharpoons}}
\providecommand{\system}{\overset{\mathcal{H}}{ \longleftrightarrow}}
	%\newcommand{\solution}[2]{\textbf{Solution:}{#1}}
\newcommand{\solution}{\noindent \textbf{Solution: }}
\newcommand{\cosec}{\,\text{cosec}\,}
\providecommand{\dec}[2]{\ensuremath{\overset{#1}{\underset{#2}{\gtrless}}}}
\newcommand{\myvec}[1]{\ensuremath{\begin{pmatrix}#1\end{pmatrix}}}
\newcommand{\mydet}[1]{\ensuremath{\begin{vmatrix}#1\end{vmatrix}}}
\numberwithin{equation}{subsection}
\makeatletter
\@addtoreset{figure}{problem}
\makeatother
\let\StandardTheFigure\thefigure
\let\vec\mathbf
\renewcommand{\thefigure}{\theproblem}
\def\putbox#1#2#3{\makebox[0in][l]{\makebox[#1][l]{}\raisebox{\baselineskip}[0in][0in]{\raisebox{#2}[0in][0in]{#3}}}}
     \def\rightbox#1{\makebox[0in][r]{#1}}
     \def\centbox#1{\makebox[0in]{#1}}
     \def\topbox#1{\raisebox{-\baselineskip}[0in][0in]{#1}}
     \def\midbox#1{\raisebox{-0.5\baselineskip}[0in][0in]{#1}}
\vspace{3cm}
\title{AI1103-Assignment 1}
\author{Kodavanti Rama Sravanth, CS20BTECH11027}
\maketitle
\newpage
\bigskip
\renewcommand{\thefigure}{\theenumi}
\renewcommand{\thetable}{\theenumi}
Download all python codes from 
\begin{lstlisting}
https://github.com/Sravanth-k27/AI1103-Assignment-1/Codes
\end{lstlisting}
%
and latex-tikz codes from 
%
\begin{lstlisting}
https://github.com/Sravanth-k27/AI1103-Assignment-1/Assignment-1.tex
\end{lstlisting}
\section*{Question(2.13):}
 A die is thrown three times. Events A and B
are defined as below:
\begin{enumerate}
 \item A : 4 on the third throw.
 \item B : 6 on the first and 5 on the second throw.
 \end{enumerate}
Find the probability of A given that B has
already occurred?
\section*{Solution(2.13):}

Let $X_i \in \{1,2,3,4,5,6\}$ where $i=1,2,3$
be the random variables representing the outcomes of throwing a die three times.\\
\begin{enumerate}
\item Probability of event A happening=Probability of $X_3=4$
\begin{align}
    \Pr{(A)}=\Pr{(X_3=4)}
\end{align}
Since all the outcomes are equally likely their probabilities are same \\
so
\begin{align}
    \Pr{(A)}=\Pr{(X_3=4)}=\frac{1}{6}
\end{align}
\item Probability of event B happening=Probability of $X_1=6,X_2=5$.\\
so
\begin{align}
    \Pr{(B)}=\Pr{(X_1=6,X_2=5)}
\end{align}
Random variable $X_1$ depends on first throw of die and random variable $X_2$ depends on second throw of die so $X_1$ and $X_2$ are independent.\\
so 
\begin{equation*}
\begin{split}
    \Pr{(X_1=6,X_2=5)}=&\Pr{(X_1=6} \Pr{(X_2=5)}\\
    &=\frac{1}{6}\times \frac{1}{6}=\frac{1}{36}
\end{split}
\end{equation*}
\begin{align}
    \Pr{(B)}=\Pr{(X_1=6,X_2=5)}=\frac{1}{36}
\end{align}
Also A,B are also independent events \\
therefore  from 0.0.2 and 0.0.4
\begin{equation*}
\begin{split}
    \Pr{(AB)}=&\Pr{(A)}  \Pr{(B)}\\
    &\frac{1}{6}\times \frac{1}{36}\\
    \end{split}
\end{equation*}
\begin{align}
    \Pr{(AB)}=\frac{1}{216}
\end{align}
Since we have to find probability of A given that B has already happened.\\
so $\Pr{(A|B)}$\\
\item By formula of conditional probability
\begin{align}
    \Pr{(A|B)}=\frac{\Pr{(AB)}}{\Pr{(B)}}
    \end{align}
    From 0.0.4 and 0.0.5 
    \begin{align}  
    \implies
    \Pr{(A|B)}=\Bigg(\frac{\frac{1}{216}}{\frac{1}{36}}\Bigg)\\
    \implies
    \Pr{(A|B)}=\frac{1}{6}
    \end{align}
    So the probability of A given that B has already happened $=\Pr{(A|B)}=\frac{1}{6}$
\end{enumerate}
\end{document}
