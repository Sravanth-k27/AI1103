\documentclass[journal,12pt,twocolumn]{IEEEtran}

\usepackage{setspace}
\usepackage{gensymb}
\singlespacing
\usepackage[cmex10]{amsmath}

\usepackage{amsthm}

\usepackage{mathrsfs}
\usepackage{txfonts}
\usepackage{stfloats}
\usepackage{bm}
\usepackage{cite}
\usepackage{cases}
\usepackage{subfig}

\usepackage{longtable}
\usepackage{multirow}

\usepackage{enumitem}
\usepackage{mathtools}
\usepackage{steinmetz}
\usepackage{tikz}
\usepackage{circuitikz}
\usepackage{verbatim}
\usepackage{tfrupee}
\usepackage[breaklinks=true]{hyperref}
\usepackage{graphicx}
\usepackage{tkz-euclide}

\usetikzlibrary{calc,math}
\usepackage{listings}
    \usepackage{color}                                            %%
    \usepackage{array}                                            %%
    \usepackage{longtable}                                        %%
    \usepackage{calc}                                             %%
    \usepackage{multirow}                                         %%
    \usepackage{hhline}                                           %%
    \usepackage{ifthen}                                           %%
    \usepackage{lscape}     
\usepackage{multicol}
\usepackage{chngcntr}

\DeclareMathOperator*{\Res}{Res}

\renewcommand\thesection{\arabic{section}}
\renewcommand\thesubsection{\thesection.\arabic{subsection}}
\renewcommand\thesubsubsection{\thesubsection.\arabic{subsubsection}}

\renewcommand\thesectiondis{\arabic{section}}
\renewcommand\thesubsectiondis{\thesectiondis.\arabic{subsection}}
\renewcommand\thesubsubsectiondis{\thesubsectiondis.\arabic{subsubsection}}


\hyphenation{op-tical net-works semi-conduc-tor}
\def\inputGnumericTable{}                                 %%

\lstset{
%language=C,
frame=single, 
breaklines=true,
columns=fullflexible
}
\begin{document}

\newcommand{\BEQA}{\begin{eqnarray}}
\newcommand{\EEQA}{\end{eqnarray}}
\newcommand{\define}{\stackrel{\triangle}{=}}
\bibliographystyle{IEEEtran}
\raggedbottom
\setlength{\parindent}{0pt}
\providecommand{\mbf}{\mathbf}
\providecommand{\pr}[1]{\ensuremath{\Pr\left(#1\right)}}
\providecommand{\qfunc}[1]{\ensuremath{Q\left(#1\right)}}
\providecommand{\sbrak}[1]{\ensuremath{{}\left[#1\right]}}
\providecommand{\lsbrak}[1]{\ensuremath{{}\left[#1\right.}}
\providecommand{\rsbrak}[1]{\ensuremath{{}\left.#1\right]}}
\providecommand{\brak}[1]{\ensuremath{\left(#1\right)}}
\providecommand{\lbrak}[1]{\ensuremath{\left(#1\right.}}
\providecommand{\rbrak}[1]{\ensuremath{\left.#1\right)}}
\providecommand{\cbrak}[1]{\ensuremath{\left\{#1\right\}}}
\providecommand{\lcbrak}[1]{\ensuremath{\left\{#1\right.}}
\providecommand{\rcbrak}[1]{\ensuremath{\left.#1\right\}}}
\theoremstyle{remark}
\newtheorem{rem}{Remark}
\newcommand{\sgn}{\mathop{\mathrm{sgn}}}
\providecommand{\abs}[1]{\vert#1\vert}
\providecommand{\res}[1]{\Res\displaylimits_{#1}} 
\providecommand{\norm}[1]{\lVert#1\rVert}
%\providecommand{\norm}[1]{\lVert#1\rVert}
\providecommand{\mtx}[1]{\mathbf{#1}}
\providecommand{\mean}[1]{E[ #1 ]}
\providecommand{\fourier}{\overset{\mathcal{F}}{ \rightleftharpoons}}
%\providecommand{\hilbert}{\overset{\mathcal{H}}{ \rightleftharpoons}}
\providecommand{\system}{\overset{\mathcal{H}}{ \longleftrightarrow}}
	%\newcommand{\solution}[2]{\textbf{Solution:}{#1}}
\newcommand{\solution}{\noindent \textbf{Solution: }}
\newcommand{\cosec}{\,\text{cosec}\,}
\providecommand{\dec}[2]{\ensuremath{\overset{#1}{\underset{#2}{\gtrless}}}}
\newcommand{\myvec}[1]{\ensuremath{\begin{pmatrix}#1\end{pmatrix}}}
\newcommand{\mydet}[1]{\ensuremath{\begin{vmatrix}#1\end{vmatrix}}}
\numberwithin{equation}{subsection}
\makeatletter
\@addtoreset{figure}{problem}
\makeatother
\let\StandardTheFigure\thefigure
\let\vec\mathbf
\renewcommand{\thefigure}{\theproblem}
\def\putbox#1#2#3{\makebox[0in][l]{\makebox[#1][l]{}\raisebox{\baselineskip}[0in][0in]{\raisebox{#2}[0in][0in]{#3}}}}
     \def\rightbox#1{\makebox[0in][r]{#1}}
     \def\centbox#1{\makebox[0in]{#1}}
     \def\topbox#1{\raisebox{-\baselineskip}[0in][0in]{#1}}
     \def\midbox#1{\raisebox{-0.5\baselineskip}[0in][0in]{#1}}
\vspace{3cm}
\title{AI1103-Assignment 2}
\author{Kodavanti Rama Sravanth, CS20BTECH11027}
\maketitle
\newpage
\bigskip
\renewcommand{\thefigure}{\theenumi}
\renewcommand{\thetable}{\theenumi}
Download all python codes from 
\begin{lstlisting}
https://github.com/Sravanth-k27/AI1103/tree/main/Assignment-2/codes
\end{lstlisting}
%
and latex-tikz codes from 
%
\begin{lstlisting}
https://github.com/Sravanth-k27/AI1103/tree/main/Assignment-2/Assignment-2.tex
\end{lstlisting}
\section*{Question(Gate EC 55):}
 Let $X_1$ be an exponential random variable
with mean 1 and $X_2$ a gamma random variable
with mean 2 and variance 2. If $X_1$ and $X_2$ are
independently distributed,then $P(X_1 < X_2)$ is equal to
\section*{Solution(Gate EC 55):}
\begin{enumerate}
    \item Given that $X_1$ is an exponential random variable. Let the P.D.F of $X_1$ be
    \begin{align}
    p_{X_1}(x_1)=
        \begin{cases}
        \lambda e^{-\lambda x_1} & x_1 \geq 0\\
        0 & x_1<0
        \end{cases}
    \end{align}
    C.D.F of $x_1$ is :
    \begin{align}
    \begin{split}
        F_{X_1}(x_1)&=\int_{-\infty}^{x_1} p_{X_1}(x_1) dx_1\\
        &=\int_{-\infty}^0 p_{X_1}(x_1) dx_1+\int_{0}^{x_1} p_{X_1}(x_1) dx_1\\
        &=\int_{-\infty}^0 0 \times dx_1+\int_{0}^ {x_1} \lambda e^{-\lambda x_1} dx_1\\
        &=1-e^{-\lambda x_1}\label{eq:0.0.2}
        \end{split}
    \end{align}
    \begin{align}
    \text{ As mean}=\lambda\\
     \text{ Given that mean}=1\\
     \text{so } \lambda=1 \label{eq:0.0.5}
    \end{align} 
     
    \item Given that $X_2$ is an gamma random variable.Let the P.D.F of $X_2$ be:
    \begin{align}
        p_{X_2}(x_2)=
        \begin{cases}
            \frac{a^b x_2^{b-1}e^{-ax_2}}{\Gamma(b)} & x_2\geq 0\\
            0 & x_2<0
        \end{cases}\label{eq:0.0.6}
    \end{align}
    \begin{align}
       \text{ Since mean}=\frac{b}{a}=2\label{eq:0.0.7}\\
         \text{Also,variance}=\frac{b}{a^2}=2\label{eq:0.0.8}
    \end{align}
  From \eqref{eq:0.0.7} and \eqref{eq:0.0.8}
  \begin{align}
      b=2,a=1\label{eq:0.0.9}
  \end{align}
  Since the total probability of $X_2$ is 1 \\
  so,\begin{align}
      \int_{-\infty}^\infty p_{X_2}(x_2) dx_2=1
      \end{align}
      \begin{align}
      \int_{-\infty} ^0 p_{X_2}(x_2)
     dx_2+\int_{0}^\infty p_{X_2}(x_2) dx_2&=1\\
      \int_{-\infty} ^0 0\times dx_2+\int_{0}^\infty \frac{a^b x_2^{b-1}e^{-ax_2}}{\Gamma(b)} dx_2&=1
      \end{align}
      \begin{align}
      \frac{a^b}{\Gamma(b)} \int_{0}^\infty x_2^{b-1}e^{-ax_2} dx_2&=1
      \end{align}
      \begin{align}
          \int_{0}^\infty x_2^{b-1}e^{-ax_2} dx_2&=\frac{\Gamma(b)}{a^b}\label{eq:0.0.14}
  \end{align}
  now substituting $a+\lambda$ for a in \eqref{eq:0.0.14} gives 
  \begin{align}
      \int_{0}^\infty x_2^{b-1}e^{-(a+\lambda)x_2} dx_2&=\frac{\Gamma(b)}{(a+\lambda)^b}\label{eq:0.0.15}
  \end{align}
  Now we have to find $P(X_1<X_2)$\\
  \item Given that $X_1$ and $X_2$ are independent random variables,so
  \begin{align}
     P(X_1<X_2|X_2)=F_{X_1}(X_2)=1-e^{-\lambda X_2}\label{eq:0.0.16}
  \end{align}
  Now,
\begin{align}
      P(X_1<X_2)&=\int_{0}^\infty F_{X_1}(X_2) \times p_{X_2}(x_2)dx_2
      \end{align}
      from \eqref{eq:0.0.6},\eqref{eq:0.0.16}
  \begin{align}
     P(X_1<X_2)=\int_{0}^\infty (1-e^{-\lambda X_2})\times \frac{a^b x_2^{b-1}e^{-ax_2}}{\Gamma(b)} dx_2
     \end{align}
     \begin{align}
     P(X_1-X_2)=\frac{a^b}{\Gamma(b)}\int_{0}^\infty x_2^{b-1}(e^{-ax_2}-e^{-(a+\lambda)x_2})dx_2
  \end{align}
      from \eqref{eq:0.0.14} and \eqref{eq:0.0.15}
      \begin{align}
     P(X_1-X_2)&=\frac{a^b}{\Gamma(b)}\brak{\frac{\Gamma(b)}{a^b}-\frac{\Gamma(b)}{(a+\lambda)^b}}
     \end{align}
 \begin{align}
      P(X_1-X_2)=1-\frac{a^b}{(a+\lambda)^b}
 \end{align}
 \begin{align}
     P(X_1-X_2)=1-\brak{\frac{a}{a+\lambda}}^b
 \end{align}
 from \eqref{eq:0.0.5} and \eqref{eq:0.0.9}
 \begin{align}
     P(X_1-X_2)&=1-\brak{\frac{1}{1+1}}^2\\
     P(X_1-X_2)&=1-\frac{1}{4}=\frac{3}{4}
 \end{align}
\end{enumerate}


\end{document}
