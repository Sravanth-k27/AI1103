\documentclass[journal,12pt,twocolumn]{IEEEtran}
\usepackage[shortlabels]{enumitem}
\usepackage{setspace}
\usepackage{gensymb}
\singlespacing
\usepackage[cmex10]{amsmath}
\usepackage{graphicx}

\usepackage{float}
\usepackage{amsthm}

\usepackage{mathrsfs}
\usepackage{txfonts}
\usepackage{stfloats}
\usepackage{bm}
\usepackage{cite}
\usepackage{cases}
\usepackage{subfig}

\usepackage{longtable}
\usepackage{multirow}

\usepackage{enumitem}
\usepackage{mathtools}
\usepackage{steinmetz}
\usepackage{tikz}
\usepackage{circuitikz}
\usepackage{verbatim}
\usepackage{tfrupee}
\usepackage[breaklinks=true]{hyperref}
\usepackage{graphicx}
\usepackage{tkz-euclide}

\usetikzlibrary{calc,math}
\usepackage{listings}
    \usepackage{color}                                            %%
    \usepackage{array}                                            %%
    \usepackage{longtable}                                        %%
    \usepackage{calc}                                             %%
    \usepackage{multirow}                                         %%
    \usepackage{hhline}                                           %%
    \usepackage{ifthen}                                           %%
    \usepackage{lscape}     
\usepackage{multicol}
\usepackage{chngcntr}

\DeclareMathOperator*{\Res}{Res}

\renewcommand\thesection{\arabic{section}}
\renewcommand\thesubsection{\thesection.\arabic{subsection}}
\renewcommand\thesubsubsection{\thesubsection.\arabic{subsubsection}}

\renewcommand\thesectiondis{\arabic{section}}
\renewcommand\thesubsectiondis{\thesectiondis.\arabic{subsection}}
\renewcommand\thesubsubsectiondis{\thesubsectiondis.\arabic{subsubsection}}


\hyphenation{op-tical net-works semi-conduc-tor}
\def\inputGnumericTable{}                                 %%

\lstset{
%language=C,
frame=single, 
breaklines=true,
columns=fullflexible
}
\begin{document}

\newcommand{\BEQA}{\begin{eqnarray}}
\newcommand{\EEQA}{\end{eqnarray}}
\newcommand{\define}{\stackrel{\triangle}{=}}
\bibliographystyle{IEEEtran}
\raggedbottom
\setlength{\parindent}{0pt}
\providecommand{\mbf}{\mathbf}
\providecommand{\pr}[1]{\ensuremath{\Pr\left(#1\right)}}
\providecommand{\qfunc}[1]{\ensuremath{Q\left(#1\right)}}
\providecommand{\sbrak}[1]{\ensuremath{{}\left[#1\right]}}
\providecommand{\lsbrak}[1]{\ensuremath{{}\left[#1\right.}}
\providecommand{\rsbrak}[1]{\ensuremath{{}\left.#1\right]}}
\providecommand{\brak}[1]{\ensuremath{\left(#1\right)}}
\providecommand{\lbrak}[1]{\ensuremath{\left(#1\right.}}
\providecommand{\rbrak}[1]{\ensuremath{\left.#1\right)}}
\providecommand{\cbrak}[1]{\ensuremath{\left\{#1\right\}}}
\providecommand{\lcbrak}[1]{\ensuremath{\left\{#1\right.}}
\providecommand{\rcbrak}[1]{\ensuremath{\left.#1\right\}}}
\theoremstyle{remark}
\newtheorem{rem}{Remark}
\newcommand{\sgn}{\mathop{\mathrm{sgn}}}
\providecommand{\abs}[1]{\vert#1\vert}
\providecommand{\res}[1]{\Res\displaylimits_{#1}} 
\providecommand{\norm}[1]{\lVert#1\rVert}
%\providecommand{\norm}[1]{\lVert#1\rVert}
\providecommand{\mtx}[1]{\mathbf{#1}}
\providecommand{\mean}[1]{E[ #1 ]}
\providecommand{\fourier}{\overset{\mathcal{F}}{ \rightleftharpoons}}
%\providecommand{\hilbert}{\overset{\mathcal{H}}{ \rightleftharpoons}}
\providecommand{\system}{\overset{\mathcal{H}}{ \longleftrightarrow}}
	%\newcommand{\solution}[2]{\textbf{Solution:}{#1}}
\newcommand{\solution}{\noindent \textbf{Solution: }}
\newcommand{\cosec}{\,\text{cosec}\,}
\providecommand{\dec}[2]{\ensuremath{\overset{#1}{\underset{#2}{\gtrless}}}}
\newcommand{\myvec}[1]{\ensuremath{\begin{pmatrix}#1\end{pmatrix}}}
\newcommand{\mydet}[1]{\ensuremath{\begin{vmatrix}#1\end{vmatrix}}}
\numberwithin{equation}{subsection}
\makeatletter
\@addtoreset{figure}{problem}
\makeatother
\let\StandardTheFigure\thefigure
\let\vec\mathbf
\renewcommand{\thefigure}{\theproblem}
\def\putbox#1#2#3{\makebox[0in][l]{\makebox[#1][l]{}\raisebox{\baselineskip}[0in][0in]{\raisebox{#2}[0in][0in]{#3}}}}
     \def\rightbox#1{\makebox[0in][r]{#1}}
     \def\centbox#1{\makebox[0in]{#1}}
     \def\topbox#1{\raisebox{-\baselineskip}[0in][0in]{#1}}
     \def\midbox#1{\raisebox{-0.5\baselineskip}[0in][0in]{#1}}
\vspace{3cm}
\title{AI1103-Assignment-4}
\author{Kodavanti Rama Sravanth,CS20BTECH11027}
\maketitle
\newpage
\bigskip
\renewcommand{\thefigure}{\theenumi}
\renewcommand{\thetable}{\theenumi}
Download all python codes from 
\begin{lstlisting}
https://github.com/Sravanth-k27/AI1103/tree/main/Assignment-4/codes
\end{lstlisting}
%
Download latex-tikz codes from 
%
\begin{lstlisting}
 https://github.com/Sravanth-k27/AI1103/tree/main/Assignment-3/Assignment-4.tex 
\end{lstlisting}
\section*{Question Gate 2021 (EC) Q.27 (EC engg section):}

 A box contains following three coins.
 \begin{enumerate}[label=\Roman*.]
 \item A coin with head on one face and tail on other face.
 \item A coin with heads on both the faces.
 \item A coin with tails on both the faces.
\end{enumerate}
A coin is picked randomly from the box and tossed .Out of the two remaining coins in the box ,one coin is then picked randomly and tossed.If the first toss results in a head,Then the probability of getting head in second toss is :
\begin{enumerate}[(A)]
\begin{multicols}{2}
\item $ \frac{2}{5}$\\
\item $\frac{1}{3}$\\
\item $ \frac{1}{2}$\\
\item $\frac{2}{3}$
\end{multicols}
\end{enumerate}
\section*{Solution ate 2021 (EC) Q.27 (EC engg section):}
Let $X_1 \in \{0,1,2\}$ be a random variable representing the coin tossed in first toss.\\
Here $X_1=0,X_1=1,X_1=2$ represents coin 1 ,coin 2,coin 3 respectively\\
Since all the coins are equally likely
\begin{align}
\pr{X_1=0}=\pr{X_1=1}=\pr{X_1=2}=\frac{1}{3}
\end{align}
Let $X_2 \in \{0,1,2\}$ be a random variable representing the coin tossed in second toss.\\
Here $X_2=0,X_2=1,X_2=2$ represents coin 1 ,coin 2,coin 3 respectively
\begin{align}
    \text{if } X_1=0: \pr{X_2=1}=\pr{X_2=2}=\frac{1}{2}\label{eq:0.0.2}\\
    \text{if } X_1=1: \pr{X_2=0}=\pr{X_2=2}=\frac{1}{2}\label{eq:0.0.3}\\
    \text{if } X_1=2: \pr{X_2=0}=\pr{X_2=1}=\frac{1}{2}\label{eq:0.0.4}
\end{align}
Let $Y,Z \in \{0,1\} $ be the random variables  which represents  outcomes of first and second toss respectively.Here 0 represents tail and 1 represents head respectively.\\
Now we have to find $\pr{Z=1|Y=1}$\\
\begin{align}
    \pr{Z=1|Y=1}=\frac{\pr{Z=1,Y=1}}{\pr{Y=1}}\label{eq:0.0.5}
\end{align}
As,
\begin{align}
   \pr{Y=1}=\sum_{i=0}^{2} \pr{Y=1|X_1=i}\times \pr{X_1=i}\label{eq:0.0.6}
\end{align}
\begin{table}[h!]
\resizebox{10cm}{!}
    {
    \begin{tabular}{|c|c|}
    \hline
    Probability & definition \\
        \hline
    $\pr{Y=1|X_1=0}=\frac{1}{2}$ & probability of getting head  on \\& \\ &  1st throw by throwing coin 1 \\
         \hline
   $\pr{Y=1|X_1=1}=1$ & probability of getting head on \\& \\& 1st throw by throwing
         coin 2\\
         \hline
    $\pr{Y=1|X_1=2}=0$ &probability of getting head on \\& \\&  1st throw by throwing
         coin 3\\
         \hline
    \end{tabular}
    }
    \caption{\label{tab:Table 1}Table 1.}
    
\end{table}
\\
from the \ref{tab:Table 1} and \eqref{eq:0.0.6}
\begin{align}
   \pr{Y=1}&=\frac{1}{2} \times \frac{1}{3}+1 \times \frac{1}{3}+ 0 \times \frac{1}{3}\\
   \pr{Y=1}&=\frac{1}{2}\label{eq:0.0.8}
\end{align}
\newpage
As,
\begin{multline}
    \pr{Z=1,Y=1}=\sum_{i=0}^{2} \pr{Z=1|X_1=i} \times \\\pr{Y-1|X_1=i} \times  \pr{X_1=i}
    \label{eq:0.0.9}
\end{multline}
\begin{table}[h!]
\resizebox{10cm}{!}
    {
    \begin{tabular}{|c|c|}
    \hline
    Probability & definition \\
        \hline
    $\pr{Z=1|X_2=0}=\frac{1}{2}=a(\text{let})$ & probability of getting head  on \\& \\ &  2nd throw by throwing coin 1\\ 
         \hline
   $\pr{Z=1|X_2=1}=1=b(\text{let})$ & probability of getting head on \\& \\& 2nd throw by throwing
         coin 2\\
         \hline
    $\pr{Z=1|X_2=2}=0=c(\text{let})$ &probability of getting head on \\& \\&  2nd throw by throwing
         coin 3\\
         \hline
    \end{tabular}
    }
    \caption{\label{tab:Table 2}Table 2.}
    \end{table}
    \\ 
    from \eqref{eq:0.0.2} and \eqref{eq:0.0.3} and \eqref{eq:0.0.4}
    \begin{align}
    \pr{Z=1|X_1=0}&=b\times \pr{X_2=1}+ c \times \pr{X_2=2}\\
        \pr{Z=1|X_1=0}&=1 \times \frac{1}{2}+ 0 \times \frac{1}{2}\\
        \pr{Z=1|X_1=0}&=\frac{1}{2}\label{eq:0.0.12}
    \end{align}
    \begin{align}
    \pr{Z=1|X_1=1}&=a \times \pr{X_2=0}+c \times \pr{X_2=2}\\
        \pr{Z=1|X_1=1}&=\frac{1}{2} \times \frac{1}{2} +0 \times \frac{1}{2}\\
        \pr{Z=1|X_1=1}&=\frac{1}{4}\label{eq:0.0.15}
    \end{align}
    \begin{align}
    \pr{Z=1|X_1=2}&=a\times \pr{X_2=0}+b \times \pr{X_2=1}\\
        \pr{Z=1|X_1=2}&=\frac{1}{2}\times \frac{1}{2}+1 \times \frac{1}{2}\\
        \pr{Z=1|X_1=2}&=\frac{3}{4}\label{eq:0.0.18}
    \end{align}
substituting \eqref{eq:0.0.12} and \eqref{eq:0.0.15} and \eqref{eq:0.0.18} in \eqref{eq:0.0.9} gives 
\begin{align}
\begin{split}
    \pr{Z=1,Y=1}&=\frac{1}{2}\times \frac{1}{2}\times \frac{1}{3}+\frac{1}{4} \times 1\times \frac{1}{3}+\frac{3}{4} \times 0 \times \frac{1}{3}\\
    \pr{Z=1,Y=1}&=\frac{1}{6}\label{eq:0.0.19}
    \end{split}
    \end{align}
    \newpage
    now substituting \eqref{eq:0.0.19} and \eqref{eq:0.0.8} in \eqref{eq:0.0.5} gives
    \begin{align}
        \pr{Z=1|Y=1}=\frac{\frac{1}{6}}{\frac{1}{2}}=\frac{1}{3}
    \end{align}
    \begin{center}
     \textbf{$\therefore$ Option B is correct}    
    \end{center}

\end{document}
