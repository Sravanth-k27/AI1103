\documentclass[journal,12pt,twocolumn]{IEEEtran}
\usepackage[shortlabels]{enumitem}
\usepackage{setspace}
\usepackage{gensymb}
\singlespacing
\usepackage[cmex10]{amsmath}
\usepackage{graphicx}

\usepackage{float}
\usepackage{amsthm}

\usepackage{mathrsfs}
\usepackage{txfonts}
\usepackage{stfloats}
\usepackage{bm}
\usepackage{cite}
\usepackage{cases}
\usepackage{subfig}

\usepackage{longtable}
\usepackage{multirow}

\usepackage{enumitem}
\usepackage{mathtools}
\usepackage{steinmetz}
\usepackage{tikz}
\usepackage{circuitikz}
\usepackage{verbatim}
\usepackage{tfrupee}
\usepackage[breaklinks=true]{hyperref}
\usepackage{graphicx}
\usepackage{tkz-euclide}

\usetikzlibrary{calc,math}
\usepackage{listings}
    \usepackage{color}                                            %%
    \usepackage{array}                                            %%
    \usepackage{longtable}                                        %%
    \usepackage{calc}                                             %%
    \usepackage{multirow}                                         %%
    \usepackage{hhline}                                           %%
    \usepackage{ifthen}                                           %%
    \usepackage{lscape}     
\usepackage{multicol}
\usepackage{chngcntr}

\DeclareMathOperator*{\Res}{Res}

\renewcommand\thesection{\arabic{section}}
\renewcommand\thesubsection{\thesection.\arabic{subsection}}
\renewcommand\thesubsubsection{\thesubsection.\arabic{subsubsection}}

\renewcommand\thesectiondis{\arabic{section}}
\renewcommand\thesubsectiondis{\thesectiondis.\arabic{subsection}}
\renewcommand\thesubsubsectiondis{\thesubsectiondis.\arabic{subsubsection}}


\hyphenation{op-tical net-works semi-conduc-tor}
\def\inputGnumericTable{}                                 %%

\lstset{
%language=C,
frame=single, 
breaklines=true,
columns=fullflexible
}
\begin{document}

\newcommand{\BEQA}{\begin{eqnarray}}
\newcommand{\EEQA}{\end{eqnarray}}
\newcommand{\define}{\stackrel{\triangle}{=}}
\bibliographystyle{IEEEtran}
\raggedbottom
\setlength{\parindent}{0pt}
\providecommand{\mbf}{\mathbf}
\providecommand{\pr}[1]{\ensuremath{\Pr\left(#1\right)}}
\providecommand{\qfunc}[1]{\ensuremath{Q\left(#1\right)}}
\providecommand{\sbrak}[1]{\ensuremath{{}\left[#1\right]}}
\providecommand{\lsbrak}[1]{\ensuremath{{}\left[#1\right.}}
\providecommand{\rsbrak}[1]{\ensuremath{{}\left.#1\right]}}
\providecommand{\brak}[1]{\ensuremath{\left(#1\right)}}
\providecommand{\lbrak}[1]{\ensuremath{\left(#1\right.}}
\providecommand{\rbrak}[1]{\ensuremath{\left.#1\right)}}
\providecommand{\cbrak}[1]{\ensuremath{\left\{#1\right\}}}
\providecommand{\lcbrak}[1]{\ensuremath{\left\{#1\right.}}
\providecommand{\rcbrak}[1]{\ensuremath{\left.#1\right\}}}
\theoremstyle{remark}
\newtheorem{rem}{Remark}
\newcommand{\sgn}{\mathop{\mathrm{sgn}}}
\providecommand{\abs}[1]{\vert#1\vert}
\providecommand{\res}[1]{\Res\displaylimits_{#1}} 
\providecommand{\norm}[1]{\lVert#1\rVert}
%\providecommand{\norm}[1]{\lVert#1\rVert}
\providecommand{\mtx}[1]{\mathbf{#1}}
\providecommand{\mean}[1]{E[ #1 ]}
\providecommand{\fourier}{\overset{\mathcal{F}}{ \rightleftharpoons}}
%\providecommand{\hilbert}{\overset{\mathcal{H}}{ \rightleftharpoons}}
\providecommand{\system}{\overset{\mathcal{H}}{ \longleftrightarrow}}
	%\newcommand{\solution}[2]{\textbf{Solution:}{#1}}
\newcommand{\solution}{\noindent \textbf{Solution: }}
\newcommand{\cosec}{\,\text{cosec}\,}
\providecommand{\dec}[2]{\ensuremath{\overset{#1}{\underset{#2}{\gtrless}}}}
\newcommand{\myvec}[1]{\ensuremath{\begin{pmatrix}#1\end{pmatrix}}}
\newcommand{\mydet}[1]{\ensuremath{\begin{vmatrix}#1\end{vmatrix}}}
\numberwithin{equation}{subsection}
\makeatletter
\@addtoreset{figure}{problem}
\makeatother
\let\StandardTheFigure\thefigure
\let\vec\mathbf
\renewcommand{\thefigure}{\theproblem}
\def\putbox#1#2#3{\makebox[0in][l]{\makebox[#1][l]{}\raisebox{\baselineskip}[0in][0in]{\raisebox{#2}[0in][0in]{#3}}}}
     \def\rightbox#1{\makebox[0in][r]{#1}}
     \def\centbox#1{\makebox[0in]{#1}}
     \def\topbox#1{\raisebox{-\baselineskip}[0in][0in]{#1}}
     \def\midbox#1{\raisebox{-0.5\baselineskip}[0in][0in]{#1}}
\vspace{3cm}
\title{AI1103-Assignment-4}
\author{Kodavanti Rama Sravanth,CS20BTECH11027}
\maketitle
\newpage
\bigskip
\renewcommand{\thefigure}{\theenumi}
\renewcommand{\thetable}{\theenumi}
Download all python codes from 
\begin{lstlisting}
https://github.com/Sravanth-k27/AI1103/tree/main/Assignment-4/codes
\end{lstlisting}
%
Download latex-tikz codes from 
%
\begin{lstlisting}
 https://github.com/Sravanth-k27/AI1103/tree/main/Assignment-4/Assignment-4.tex 
\end{lstlisting}
\section*{Question Gate 2021 (EC) Q.27 (EC engg section):}

 A box contains following three coins.
 \begin{enumerate}[label=\Roman*.]
 \item A coin with head on one face and tail on other face.
 \item A coin with heads on both the faces.
 \item A coin with tails on both the faces.
\end{enumerate}
A coin is picked randomly from the box and tossed .Out of the two remaining coins in the box ,one coin is then picked randomly and tossed.If the first toss results in a head,Then the probability of getting head in second toss is :
\begin{enumerate}[(A)]
\begin{multicols}{2}
\item $ \frac{2}{5}$\\
\item $\frac{1}{3}$\\
\item $ \frac{1}{2}$\\
\item $\frac{2}{3}$
\end{multicols}
\end{enumerate}
\section*{Solution Gate 2021 (EC) Q.27 (EC engg section):}
Let $X \in \{1,2\}$ be a random variable.\\
\\Let  $Y \in \{1,2,3\}$ be a random variable.\\
\\Let $Z \in \{0,1\}$ be a random variable.\\
\newpage
\begin{table}[h!]
    \resizebox{7cm}{!}
    {
    \begin{tabular}{|c|c|}
    \hline
        Event & Definition \\
         \hline
         $X=1$ & Represents trail 1\\&\\
         \hline
         $X=2$ & Represents trail 2\\&\\
         \hline
         $Y=1$ & selecting coin 1 for a trail\\&\\
         \hline
         $Y=2$ & selecting coin 2 for a trail\\&\\
         \hline
         $Y=3$ & selecting coin 3 for a trail\\&\\
         \hline
         $Z=0$ & getting tail on a trail\\&\\
         \hline
         $Z=1$ & getting head on a trail\\&\\
         \hline 
         $X=1,Z=1|Y=1$ & getting head on first trail
         \\ & by tossing coin 1\\
         \hline
         $X=1,Z=1|Y=2$ & getting head on first trail
         \\ & by tossing coin2 \\
         \hline
         $X=1,Z=1|Y=3$ & getting head on first trail
         \\ & by tossing coin3 \\
         \hline
         $X=1,Z=1$ & getting head on first trail\\&\\
         \hline
         $X=1,X=2,Z=1$ & getting head on both \\ &first and second trails\\
         \hline
    \end{tabular}
    }
    \caption{\label{tab:table-1}TABLE-1.}
\end{table}
Now we need to find\\ $\pr{X=2,Z=1|X=1,Z=1}$= a ( let )\\
\\From conditional probability we have
\begin{align}
    a=\frac{\pr{X=1,X=2,Z=1}}{\pr{X=1,Z=1}}
\end{align}
\begin{multline}
    \pr{X=1,Z=1}=\\
    \sum_{i=1}^{3}\pr{X=1,Z=1|Y=i}
    \times \pr{Y=i}
\end{multline}
\newpage
\begin{table}[h!]
\resizebox{7cm}{!}
{
    \begin{tabular}{|c|c|}
         \hline
         Probability & Value\\
         \hline
         $\pr{Y=1}$ & $\frac{1}{3}$\\&\\
         \hline
         $\pr{Y=2}$ & $\frac{1}{3}$\\&\\
         \hline
         $\pr{Y=3}$ & $\frac{1}{3}$\\&\\
         \hline
         $\pr{X=1,Z=1|Y=1}$ & $\frac{1}{2}$ \\&\\
         \hline
         $\pr{X=1,Z=1|Y=2}$ & 1\\&\\
         \hline
         $\pr{X=1,Z=1|Y=3}$  & 0 \\&\\
         \hline
         $\pr{X=1,Z=1}$ & $=\frac{1}{3}\times \frac{1}{2}+\frac{1}{3}\times 1+\frac{1}{3} \times 0$\\&\\ &$=\frac{1}{2}$\\&\\
         \hline
         $\pr{X=1,X=2,Z=1}$ & $=\frac{1}{3}\times \frac{1}{2}\times \frac{1}{2}+\frac{1}{3}\times \frac{1}{2}\times \frac{1}{2}+\frac{1}{3}\times 0$\\&\\&$=\frac{1}{6}$\\&\\
         \hline
    \end{tabular}
    }
    \caption{\label{tab:Table-2}Table-2.}
\end{table}
from \ref{tab:Table-2} 
\begin{align}
\pr{X=2,Z=1|X=2,Z=1}&=\frac{\frac{1}{6}}{\frac{1}{2}}
\end{align}
\begin{align}
\pr{X=2,Z=1|X=2,Z=1}&=\frac{1}{3}
\end{align}
\\Hence the required probability is $\frac{1}{3}$\\
 \\ \textbf{$\therefore$ Option B is correct}

\end{document}
