\documentclass[journal,12pt,twocolumn]{IEEEtran}
\usepackage[shortlabels]{enumitem}
\usepackage{setspace}
\usepackage{gensymb}
\singlespacing
\usepackage[cmex10]{amsmath}
\usepackage{graphicx}

\usepackage{float}
\usepackage{amsthm}

\usepackage{mathrsfs}
\usepackage{txfonts}
\usepackage{stfloats}
\usepackage{bm}
\usepackage{cite}
\usepackage{cases}
\usepackage{subfig}

\usepackage{longtable}
\usepackage{multirow}

\usepackage{enumitem}
\usepackage{mathtools}
\usepackage{steinmetz}
\usepackage{tikz}
\usepackage{circuitikz}
\usepackage{verbatim}
\usepackage{tfrupee}
\usepackage[breaklinks=true]{hyperref}
\usepackage{graphicx}
\usepackage{tkz-euclide}

\usetikzlibrary{calc,math}
\usepackage{listings}
    \usepackage{color}                                            %%
    \usepackage{array}                                            %%
    \usepackage{longtable}                                        %%
    \usepackage{calc}                                             %%
    \usepackage{multirow}                                         %%
    \usepackage{hhline}                                           %%
    \usepackage{ifthen}                                           %%
    \usepackage{lscape}     
\usepackage{multicol}
\usepackage{chngcntr}

\DeclareMathOperator*{\Res}{Res}

\renewcommand\thesection{\arabic{section}}
\renewcommand\thesubsection{\thesection.\arabic{subsection}}
\renewcommand\thesubsubsection{\thesubsection.\arabic{subsubsection}}

\renewcommand\thesectiondis{\arabic{section}}
\renewcommand\thesubsectiondis{\thesectiondis.\arabic{subsection}}
\renewcommand\thesubsubsectiondis{\thesubsectiondis.\arabic{subsubsection}}


\hyphenation{op-tical net-works semi-conduc-tor}
\def\inputGnumericTable{}                                 %%

\lstset{
%language=C,
frame=single, 
breaklines=true,
columns=fullflexible
}
\begin{document}

\newcommand{\BEQA}{\begin{eqnarray}}
\newcommand{\EEQA}{\end{eqnarray}}
\newcommand{\define}{\stackrel{\triangle}{=}}
\bibliographystyle{IEEEtran}
\raggedbottom
\setlength{\parindent}{0pt}
\providecommand{\mbf}{\mathbf}
\providecommand{\pr}[1]{\ensuremath{\Pr\left(#1\right)}}
\providecommand{\qfunc}[1]{\ensuremath{Q\left(#1\right)}}
\providecommand{\sbrak}[1]{\ensuremath{{}\left[#1\right]}}
\providecommand{\lsbrak}[1]{\ensuremath{{}\left[#1\right.}}
\providecommand{\rsbrak}[1]{\ensuremath{{}\left.#1\right]}}
\providecommand{\brak}[1]{\ensuremath{\left(#1\right)}}
\providecommand{\lbrak}[1]{\ensuremath{\left(#1\right.}}
\providecommand{\rbrak}[1]{\ensuremath{\left.#1\right)}}
\providecommand{\cbrak}[1]{\ensuremath{\left\{#1\right\}}}
\providecommand{\lcbrak}[1]{\ensuremath{\left\{#1\right.}}
\providecommand{\rcbrak}[1]{\ensuremath{\left.#1\right\}}}
\theoremstyle{remark}
\newtheorem{rem}{Remark}
\newcommand{\sgn}{\mathop{\mathrm{sgn}}}
\providecommand{\abs}[1]{\vert#1\vert}
\providecommand{\res}[1]{\Res\displaylimits_{#1}} 
\providecommand{\norm}[1]{\lVert#1\rVert}
%\providecommand{\norm}[1]{\lVert#1\rVert}
\providecommand{\mtx}[1]{\mathbf{#1}}
\providecommand{\mean}[1]{E[ #1 ]}
\providecommand{\fourier}{\overset{\mathcal{F}}{ \rightleftharpoons}}
%\providecommand{\hilbert}{\overset{\mathcal{H}}{ \rightleftharpoons}}
\providecommand{\system}{\overset{\mathcal{H}}{ \longleftrightarrow}}
	%\newcommand{\solution}[2]{\textbf{Solution:}{#1}}
\newcommand{\solution}{\noindent \textbf{Solution: }}
\newcommand{\cosec}{\,\text{cosec}\,}
\providecommand{\dec}[2]{\ensuremath{\overset{#1}{\underset{#2}{\gtrless}}}}
\newcommand{\myvec}[1]{\ensuremath{\begin{pmatrix}#1\end{pmatrix}}}
\newcommand{\mydet}[1]{\ensuremath{\begin{vmatrix}#1\end{vmatrix}}}
\numberwithin{equation}{subsection}
\makeatletter
\@addtoreset{figure}{problem}
\makeatother
\let\StandardTheFigure\thefigure
\let\vec\mathbf
\renewcommand{\thefigure}{\theproblem}
\def\putbox#1#2#3{\makebox[0in][l]{\makebox[#1][l]{}\raisebox{\baselineskip}[0in][0in]{\raisebox{#2}[0in][0in]{#3}}}}
     \def\rightbox#1{\makebox[0in][r]{#1}}
     \def\centbox#1{\makebox[0in]{#1}}
     \def\topbox#1{\raisebox{-\baselineskip}[0in][0in]{#1}}
     \def\midbox#1{\raisebox{-0.5\baselineskip}[0in][0in]{#1}}
\vspace{3cm}
\title{AI1103-Assignment-5}
\author{Kodavanti Rama Sravanth,CS20BTECH11027}
\maketitle
\newpage
\bigskip
\renewcommand{\thefigure}{\theenumi}
\renewcommand{\thetable}{\theenumi}
Download all python codes from 
\begin{lstlisting}
https://github.com/Sravanth-k27/AI1103/tree/main/Assignment-5/codes
\end{lstlisting}
%
Download latex-tikz codes from 
%
\begin{lstlisting}
 https://github.com/Sravanth-k27/AI1103/tree/main/Assignment-5/Assignment-5.tex 
\end{lstlisting}
\section*{UGC/math 2018(june math set a )Q.18:}
Two students are solving the same problem independently,if the probability of first one solves the problem is $\frac{3}{5}$ and the probability that the second one solves the problem is $\frac{4}{5}$ ,what is the probability that atleast one of them solves the problem?
\begin{enumerate}[(A)]
\begin{multicols}{2}
\item $ \frac{17}{25}$\\
\item $\frac{19}{25}$\\
\item $ \frac{21}{25}$\\
\item $\frac{23}{25}$
\end{multicols}
\end{enumerate}
\section*{Solution UGC/math 2018(june math set a )Q.18::}
Let X,Y be two events representing solving the problem by students A,B respectively.\\
Given 
\begin{align}
    \pr{X}=\frac{3}{5}\label{eq:0.0.1}\\
    \pr{Y}=\frac{4}{5}\label{eq:0.0.2}
\end{align}
Since students solve the problem independently,So events X and Y are independent,
For independent events
\begin{align}
    \pr{XY}&=\pr{X}\times \pr{Y}
\intertext{from \eqref{eq:0.0.1} and \eqref{eq:0.0.2}}
    \pr{XY}&=\frac{3}{5}\times \frac{4}{5}\\
    &=\frac{12}{25}\label{eq:0.0.5}
\end{align}
\\
\\Now we have to find probability of solving the problem by atleast one of them i.e $\pr{X+Y}$.\\
As,
\begin{align}
    \pr{X+Y}&=\pr{X}+\pr{Y}-\pr{XY}\\
\intertext{from \eqref{eq:0.0.1}, \eqref{eq:0.0.2}, \eqref{eq:0.0.5}}
&=\frac{3}{5}+\frac{4}{5}-\frac{12}{25}\\
&=\frac{23}{25}
\end{align}
Hence the required probability is $\frac{23}{25}$\\
\\    \textbf{$\therefore$ Option D is correct} 
\end{document}
